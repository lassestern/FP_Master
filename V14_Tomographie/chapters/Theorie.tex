\section{Ziel}
    Ziel dieses Versuches ist es die elementspezifische räumliche Zusammensetzung eines Objektes zu bestimmen.Dazu wird die Methodik der Gamma-Tomographie genutzt. Bei dieser werden entlang mehrerer räumlicher
    Achsen des Objekts Absorptionsmessungen mit Gamma-Strahlung durchgeführt, die in Kombination auf die gesuchte elementspezifische räumliche Zusammensetzung schließen lassen.   
    verschiedener räumlicher  
\section{Theoretische Grundlagen}
    \subsection{Gamma-Strahlen-Quellen}
        Für die notwendigen Absorptionsmessungen muss zunächst Gamma-Strahlung erzeugt werden. Gamma-Strahlung beschreibt Photonen mit einer Energie über \SI{200}{\kilo\electronvolt} und kann auch verschiedenen
        Wegen entstehen. Hier soll die Entstehung bei radioaktiven Zerfällen betrachtet werden. Explizit werden die $\beta^-$-Zerfälle von \ce{^{137}Cs} und \ce{^{60}Co} betrachtet. Diese Elemente zerfallen 
        zunächst in angeregte Zustände eines weiteren Elements und gehen dann unter Aussendung eines Photons in dessen Grundzustand über. Wie in Abbildung ... zu sehen, kann \ce{^{137}Cs} nur in einen 
        angeregten Zustand von \ce{^{137}Ba} zerfallen. Bei dessen Übergang in den Grundzustand $ \ce{^{137}Ba}^* \rightarrow \ce{^{137}Ba} + \gamma$ wird ein Photon der Energie \SI{661.7}{\kilo\electronvolt}
        ausgesendet. Demnach strahlt eine ein \ce{^{137}Cs} mit einer maximalen Intensität bei der angegebenen Energie von \SI{661.7}{\kilo\electronvolt}.

        \FloatBarrier

        \begin{figure}[h]
          \centering
          \includegraphics[width = 1\textwidth]{pictures/cs.png}
          \caption{Die möglichen $\beta^-$-Zerfälle von \ce{^{137}Cs} in \ce{^{137}Ba} sowie dessen angeregten Zustand \ce{^{137}Ba}$^*$ und anschließender Übergang in den Grundzustand von \ce{^{137}Ba} unter Aussendung eines Photons. Bearbeitet aus \cite{stolz_radioaktivitat_2003}}
          \label{fig:cs_schema}
        \end{figure}
    
        \FloatBarrier
        
        Für den in Abbildung ... skizzierten Zerfall von \ce{^{60}Co} sind Übergänge in zwei verschiedene angeregte Zustände von \ce{^{60}Ni} möglich. Der energetisch höhere Zustand liegt bei 
        \SI{2505.7}{\kilo\electronvolt} und der niedrigere bei \SI{1332.5}{\kilo\electronvolt}. Der energetisch niedrigere Zustand geht direkt in den Grundzustand über und es wird ein Photon mit einer Energie
        von \SI{1332.5}{\kilo\electronvolt} ausgesendet. Die Relaxation des energetisch höheren Zustands findet in zwei Schritten statt. Zunächst geht dieser Zustand in den niederenergetischen angeregten Zustand 
        über, wobei ein Photon mit der Energie \SI{1173.2}{\kilo\electronvolt} ausgesendet wird. Anschließend geht es in den Grundzustand des \ce{^{60}Ni} über. Aufgrund der zwei angeregten Endzustände des 
        $\beta^-$-Zerfalls strahlt eine \ce{^{60}Co}-Quzelle mit zwei charakteristischen Energien.

        \FloatBarrier

        \begin{figure}[h]
          \centering
          \includegraphics[width = 1\textwidth]{pictures/co.png}
          \caption{Die möglichen $\beta^-$-Zerfälle von \ce{^{60}Co} in die angeregten Zustände von \ce{^{60}Ni} und anschließende Übergänge in den Grundzustand von \ce{^{60}Ba} unter Aussendung zwei Photonen verschiedener Energien. Bearbeitet aus \cite{stolz_radioaktivitat_2003}}
          \label{fig:co_schema}
        \end{figure}
    
        \FloatBarrier

      \subsection{Absorption von Photonen}
        Die Absorption von Photonen wird über die Änderung der Intensität $I$ einer Strahlungsquelle über das \textbf{Lambert-Beerśche-Gesetz} beschrieben

        \begin{equation}
          I(x) = I_0 \exp\left(-\mu x\right),
        \label{eqn:absorptionsgesetz}
        \end{equation}

        das die Intensität $I$ in einer Entfernung $x$ von einem Ausgangspunkt mit der Ausgangsintensität $I_0$ in Abhängigkeit der Entfernung und des Absorptionskoefizientens $\mu$ des Ausbreitungsmediums 
        angibt. Der gesamte Absorptionskoefizient $\mu$ ist die Summe der Absorptionskoefizienten vieler Prozesse
        
        \begin{equation*}
          \mu = \mu_{PE} + \mu_{CS} + \mu_{PP} + \mu_{RS},
        \end{equation*}

        wie der Photoemission (PE), der Compton-Streuung (CS), der Paar-Produktion (PP) und der Rayleigh-Streuung (RS). Der gesamte Absorptionskoefizient $\mu$ ist zum einen von der Photonenenergie $E$ und 
        vom Ausbreitungsmedium abhängig. In Abbildung \ref{fig:absorptionskoeffizient} ist der energieabhängige Verlauf des gesamten Absorptionskoefizientens sowie der der hauptsächlich beitragenden 
        Absorptionskoefizienten der Paar-Produktion, Photoemission und Compton-Streuung für Germanium dargestellt. 

        \FloatBarrier

        \begin{figure}[h]
          \centering
          \includegraphics[width = 0.6\textwidth]{pictures/absorptionskoeffizient.png}
          \caption{Der energieabhängige Verlauf des Absorptionskoefizientens von Germanium und dessen Bestandteilen der einzelnen Prozesse (PP, PE, CS).Entnommen aus \cite{gilmore_practical_2008}}
          \label{fig:absorptionskoeffizient}
        \end{figure}
    
        \FloatBarrier

      \newpage
      \subsection{Intensitätsmessung}
        Um die Intensität der Strahlung in Abhängigkeit der Energie zu messen, wird von einem \textbf{Szintillationsdetektor} in Kombination mit einem \textbf{Diskrimantor} und einem 
        \textbf{Vielkanal-Analysator} Gebrauch gemacht.

        \subsubsection*{Szintillationsdetektor}
          Das Konzept von Szintillatoren beruht darauf, dass einfallende Strahlung hoher Energie Atome des Szintillationsmediums entweder ionisiert oder anregt und 
          diese beim Relaxieren optische Photonen freisetzen. Die Menge an freigesetzten Photonen hängt dabei von der Energie der einfallenden Strahlung ab. Die optischen Photonen des Relaxationsprozesses 
          werden anschließend von Photomultipliern detektiert. %Grundlegend wird zwischen organischen und anorganischen Szintillatoren unterschieden.

        \subsubsection*{Diskriminatoren}
          Um nur optische Photonen aus dem Szintillationsdetektor zu detektieren wird ein Diskrimantor eingesetzt. Dieser gibt erst ab einem einen Schwellwert übersteigenden Eingangssignal ein Ausgangssignal 
          aus. So kann verhindert werden, dass bereits ein einzelnes spontan emmitiertes Photon einen Ausgangsimpuls am Photomultiplier hervorruft, der fälschlicherweise auf ein optisches Photon des 
          Szintillationsdetektors zurückgeführt werden würde.

        \subsubsection*{Vielkanal-Analysator}
          Aus dem Photomultiplier erreichen den Vielkanal-Analysator elektrische Signale, deren Stärke proportional zur Energie der Strahlung, die ein Szintillationselektron anregt, angenommen werden kann. 
          Der Vielkanal-Analysator besitzt nun einen digitalen Speicher, der für die verschiedenen Impulstärken verschiedene Speicherplätze besitzt. Die Impulsstärken und zugehörigen Speicherplätze werden
          durch eine Kalibrierung Photonenenergien der die Szintillationselektronen anregenden Strahlung zugeordnet. Durch die Einsortierung der eingehenden Impulse in die verschiedenen Speicherplätze, kann so 
          ein Histogramm erstellt werden, dass die energieabhängige Intensität der auf den Detektor treffenden Intensität darstellt. 

           

