\newpage
\section{Diskussion}
    Die Aufspaltung der Energieniveaus konnte in diesem Versuch für die Cadmium-Übergänge der Wellenlängen \SI{480.0}{\nano\metre} und \SI{643.8}{\nano\metre} für den $\sigma$-polarisierten Anteil des Lichts 
    beobachtet werden. Die Aufspaltung des $\pi$-polarisierten Anteils des Lichts des \SI{480.0}{\nano\metre}-Übergangs konnte in diesem Versuch nicht beobachtet werden, sodass externe Daten verwendet werden 
    mussten. Dies könnte daran liegen, dass der Elektromagnet nach einiger Zeit aufgrund von Erhitzung sehr instabil wurde und nur schwer hohe Magnetfeldstärken erzeugt werden konnten. Insgesamt führte das
    Aufheizen des Elektromagneten zu Schwierigkeiten beim Erzeugen hoher Magnetfeldstärken.\newline 
    In den aufgenommenen Interferenzmustern sind die Maxima und die Verschiebung dieser eindeutig zu erkennen. Da die 
    Maxima per Auge identifiziert werden, ist der Abstand der etwas verwaschenen Maxima dennoch fehlerbehaftet und trägt vermutlich stark zu den Fehlern in der Bestimmung der Landé-Faktoren bei. 
    Aufgrund der hohen Zoomstufe der Kamera und des digitalen Bildbearbeitungsprogramms wurde der Ablesefehler dennoch auf nur einen Pixel festgelegt.\newline
    Der Landé-Faktor für den $\sigma$-polarisierten Anteil des
    \SI{643.8}{\nano\metre}-Übergangs weicht um circa einen Prozent vom Literaturwert und kann damit als eindeutig bestimmt bezeichnet werden. Für den \SI{480.0}{\nano\metre}-Übergangtreten mit \num{24.6}\% 
    ($g_{\text{blau}, \sigma}$) und \num{32.6}\% ($g_{\text{blau}, \sigma}$) deutlich stärkere Abweichungen auf. Ursächlich dafür kann die stärkere Verschmierung der Maxima im Interferenzmuster des 
    $\pi$-polarisierten Anteils des \SI{480.0}{\nano\metre}-Übergangs sein. Zusätzlich ist der Landé-Faktor des $\sigma$-polarisierten Anteil des \SI{480.0}{\nano\metre}-Übergangs ein Mittelwert aus zwei 
    möglichen Übergängen, die das beobachten eines einzelnen Übergangs unmöglich machen. 


    \begin{table}[h]
        \centering
        \caption{Die bestimmten Landé-Faktoren $\text{g}_{\text{exp}}$ für die verschiedenen Übergänge sowie die zugehörigen Literaturwerte $\text{g}_{\text{Lit}}$, die dazu relative Abweichung A und die angelegten Magnetfeldstärken zur Vermessung der Aufspaltung.}
        \label{tab:blau_pi}
    
        \begin{tabular}{c c c c c}
          \toprule
          {Übergang} & {$B$ [mT]} & {$\text{g}_{\text{exp}}$} & {$\text{g}_{\text{Lit}}$} & {A} \\ 
          \midrule
           \SI{643.8}{\nano\metre} - $\sigma$  & 464  & $0.993\pm0.013$  &   1.00      &  $(0.7\pm1.3$)\% \\
           \SI{480.0}{\nano\metre} - $\sigma$ & 263  & $2.180\pm0.040$  &   1.75      &  $(24.6\pm2.1$)\% \\
           \SI{480.0}{\nano\metre} - $\pi$    & 1009 & $0.663\pm0.004$  &   0.50      &  $(32.6\pm0.8$)\% \\

          \bottomrule
        \end{tabular}
      \end{table}
      \FloatBarrier
    
    
