    \section{Ziel}
        Ziel dieses Versuches ist es den Zeemann-Effekt an den Spektrallinien zu untersuchen und
        daraus die Landé-Faktoren zu bestimmen.
    \section{Theoretische Grundlagen}
        \subsection{Das magnetische Moment eines Elektrons}
            Hüllenelektronen in einem Atom besitzen zwei Drehimpulse - den Bahndrehimpuls $\vec{l}$ und den Spin $\vec{s}$.
            Für die entsprechenden magnetischen Momente gilt
            \begin{equation}
                \vec{\mu}_l = -\frac{\mu_B}{\hbar}\vec{l} \quad\text{und}\quad \vec{\mu}_s = -g_s\frac{\mu_B}{\hbar}\vec{s} \, ,
            \end{equation}
            mit dem Bor'schen Magneton
            \begin{equation}
                \mu_B=-\frac{e_0\hbar}{2m_0}
            \end{equation}
            und dem sogennanten Landé-Faktor $g_s$.
            Dabei ergeben sich die Beträge der Drehimpulse aus den dazugehörigen Quantenzahlen $l$ und $s$
            \begin{equation}
                \vert\vec{l}\vert=\sqrt{l(l+1)}\cdot\hbar \quad\text{und}\quad \vert\vec{s}\vert=\sqrt{s(s+1)}\cdot\hbar
            \end{equation}
        \subsection{Die Spin-Bahn Wechselwirkung}
            Das magnetische Moment $\vec{\mu}_s$ koppelt an das B-Feld, welches aus der Bewegung des Atomkerns,
            aus der Sicht des Elektrons entsteht.
            Bei einem Sytem mit mehreren Elektronen werden dabei folgende zwei Grenzfälle behandelt.
            \subsubsection{LS-Kopplung}
                Im Falle von Atomen mit kleinen Ordnungszahlen überwiegt die Wechselwirkung zwischen
                den Bahndrehimpulsen und den Spinmomenten untereinander gegenüber der Spin-Bahn Wechselwirkung der einzelnen Elektronen.
                Die Drehimpulse der einzelnen Elektronen werden dabei jeweils zu dem Gesamtbahndrehimpuls und dem Gesamtspin
                \begin{align}
                    \vec{L}&=\sum\vec{l}_i \qquad\text{mit}\quad \vert\vec{L}\vert=\sqrt{L(L+1)}\cdot\hbar \\
                    \vec{S}&=\sum\vec{s}_i \qquad\text{mit}\quad \vert\vec{S}\vert=\sqrt{S(S+1)}\cdot\hbar
                \end{align}
                addiert, wobei $L$ und $S$ die Quantenzahlen beschreiben.
                Diese ergeben zusammen wiederrum den Gesamtdrehimpuls
                \begin{equation}
                    \vec{J}=\vec{L}+\vec{S} \qquad\text{mit}\quad \vert\vec{J}\vert=\sqrt{J(J+1)}\cdot\hbar \, .
                \end{equation}
            \subsubsection{jj-Kopplung}
                Bei Atomen mit großen Ordnungszahlen $Z$ überwiegt die Spin-Bahn Kopplung der Elektronen der Wechselwirkung untereinander,
                da diese proportional zu $Z$ ist.
                Für jedes Elektron ergibt sich jeweils der Gesamtdrehimpuls
                \begin{equation}
                    \vec{j}_i=\vec{l}_i+\vec{s}_i 
                \end{equation}
                und für das gesamte Atom gilt
                \begin{equation}
                    \vec{J}=\sum\vec{j}_i \, .
                \end{equation}
            Im Rahmen dieses Versuches befinden wir uns aber im Grenzfall der LS-Kopplung.
        \subsection{Aufspaltung der Energieniveas im Magnetfeld}
            Wird ein externes Magnetfeld angelegt, so spalten sich die Energieniveaus in Abhängigkeit der Gesamtquantenzahl $J$ auf.
            Es ergeben sich dabei $2J+1$ äquidistante Energieniveaus mit Energien von
            \begin{equation}
                E_{B\neq 0}=E_0+mg_j\mu_BB \qquad\text{mit}\quad m=-J,-J+1,...,J-1,J \, .
                \label{eqn:E}
            \end{equation}
            Für den Landé-Faktor $g_j$ gilt
            \begin{equation}
                g_j=\frac{3J(J+1)+S(S+1)-L(L+1)}{2J(J+1} \, .
            \end{equation}
            Diese Aufspaltung ist in Abbildung \ref{fig:Aufspaltung} graphisch dargestellt.
            \begin{figure}[h]
                \centering
                \includegraphics[width = 0.7\textwidth]{pictures/aufspaltung.png}
                \caption{Aufspaltung von Energieniveaus bei angelegtem Magnetfeld für $J=2$. Entnommen aus \cite{haken_atom-_2004}}
                \label{fig:Aufspaltung}
            \end{figure}
        \subsection{Zeemann-Effekt}
            Die Aufspaltung von angeregten Energieniveaus hat eine Aufspaltung der Spektrallinien als Folge,
            da zusätzliche Übergänge zwischen den Niveaus möglich werden.
            Diese Spektrallinienaufspaltung wird als Zeemann-Effekt bezeichnet.
            Die Übergänge zwischen den verschiedenen Energieniveaus können dabei nur dann stattfinden, wenn $\Delta m=0,\pm 1$.
            Bei $\Delta m=0$ wird dabei in Richtung von $\vec{B}$ linearpolarisiertes Licht ($\pi$-Licht) emittiert.
            Für $\Delta m=\pm 1$ ist das ausgesendete Licht ($\sigma$-Licht) zirkular um $\vec{B}$ polarisiert.
            Die Aufspaltung der Zeemann-Linien $\Delta E$ lässt sich mit der Formel
            \begin{equation}
                \Delta E= \left(m_\uparrow g_\uparrow-m_\downarrow g_\downarrow \right)\mu_BB
            \end{equation}
            berechnen.
    
            Man unterscheidet zwischen dem sogennanten normalen und dem anormalen Zeemann-Effekt.
            \subsubsection{Normaler Zeemann-Effekt}
                Dieser tritt auf bei einem Übergang von zwei Zuständen mit $S=0$, der Spin wird also nicht berücksichtigt.
                Für den Landé-Faktor gilt dabei $g_j=1$ und mit Formel \ref{eqn:E}
                ergibt sich für die Abstände der aufgespaltenen Energieniveaus untereinander jeweils
                \begin{equation}
                    E_A= \mu_BB \, .
                \end{equation}
                Diese Zusammenhänge sind in Abbildung \ref{fig:NZ} anhand des $1P_1\leftrightarrow 1D_2$ Übergangs einer Cd-Lampe zu sehen.
                \begin{figure}[h]
                    \centering
                    \includegraphics[width = 0.7\textwidth]{pictures/normal.png}
                    \caption{Normaler Zeemann-Effekt beim $1P_1\leftrightarrow 1D_2$ Übergang.}
                    \label{fig:NZ}
                \end{figure}
                Hier ergibt sich für den normalen Zeemann-Effekt eine Aufspaltung in genau drei Spektrallinien.
                einer Aufspaltung $\Delta E$ der Zeemann-Linien von
                \begin{align}
                    (1)(2)(3): \Delta E&=\mu_BB \\
                    (4)(5)(6): \Delta E&=0 \\
                    (7)(8)(9): \Delta E&=-\mu_BB \, .
                \end{align}
                Durch Beobachtung in Feldrichtung ist das linear polarisierte $\pi$-Licht nicht sichtbar und es
                kann nur der longitudinale Anteil des zirkular polarisierten $\sigma$-Licht identifiziert werden.
                Bei longitudinaler Beobachtung beobachtet man also nur eine Aufspaltung in zwei Spektrallinien (siehe Abbildung \ref{fig:Beob})
                \begin{figure}[h]
                    \centering
                    \includegraphics[width = 0.4\textwidth]{pictures/beobachtung.png}
                    \caption{Aufspaltung beim normalen Zeemann-Effekt aus verschiedenen Beobachtungsrichtungen.Entnommen aus \cite{haken_atom-_2004}}
                    \label{fig:Beob}
                \end{figure}
            \subsubsection{Anormaler Zeemann-Effekt}
                In diesem Falle wird der Spin $S\neq 0$ berücksichtigt.
                Daraus ergeben sich Landé-Faktoren $g\neq 1$ und nach Formel \ref{eqn:E}
                existiert für die zwei Energieniveaus zwischen welchen die Übergänge stattfinden, eine spinabhängige Aufspaltung.
                Die äquidistant aufgespaltenen Energieniveaus haben untereinander einen Abstand von jeweils
                \begin{equation}
                    E_A= g_{\uparrow\text{bzw}\downarrow}\mu_BB \, .
                \end{equation}
                Das führt im Endeffekt zu einer deutlich linienreicheren Spektrallinienaufspaltung im Falle eines Übergangs zwischen den Energienieveaus. 
                In Abbildung \ref{fig:ANZ} ist der anormale Zeemann-Effekt anhand des $3S_1\leftrightarrow 3P_1$ Übergangs einer Cd-Lampe verdeutlicht.
                \begin{figure}[h]
                    \centering
                    \includegraphics[width = 0.7\textwidth]{pictures/anormal.png}
                    \caption{Anormaler Zeemann-Effekt beim $3S_1\leftrightarrow 3P_1$ Übergang.}
                    \label{fig:ANZ}
                \end{figure}
                Hier ergibt sich für die Energiedifferenzen:
                \begin{align}
                    (1): \Delta E&=\frac{3}{2}\mu_BB \\
                    (2): \Delta E&=2\mu_BB \\
                    (3): \Delta E&=-\frac{1}{2}\mu_BB \\
                    (4): \Delta E&=0 \\
                    (5): \Delta E&=\frac{1}{2}\mu_BB \\
                    (6): \Delta E&=-2\mu_BB \\
                    (7): \Delta E&=-\frac{3}{2}\mu_BB \\                
                \end{align}


