\newpage
\section{Daten}
    \begin{table}[h]
        \centering
        \caption{Hier sind die Nulldurchgänge, die der Anzahl der Maxima entsprechen, bei einer Drehung der Spiegel um 10° für 10 Messungen aufgeführt.}
        \label{tab:Nullen_Glas}
        \begin{tabular}{c c}
        \toprule
        {Messung} & {Maxima}  \\
        \midrule
        1	 &  33  \\
        2	 &  35  \\
        3	 &  33  \\
        4	 &  33  \\
        5	 &  34  \\
        6	 &  33  \\
        7	 &  33  \\
        8	 &  33  \\
        9	 &  33  \\
        10	 &  34  \\
        
                      
        \bottomrule
        \end{tabular}
    \end{table}

    \FloatBarrier

    \begin{table}[h]
        \centering
        \caption{Hier sind die Nulldurchgänge, die der Anzahl der Maxima entsprechen, bei einer Erhöhung des Drucks von 6 mbar zu 1006 mbar in 50 mbar-Schritten aufgetragen. Da alle drei Messungen exakt gleich waren, wird nur eine Messung aufgetragen.}
        \label{tab:Nullen_Luft}
        \begin{tabular}{c c}
        \toprule
        {Druck [mbar]} & {Maxima} \\
        \midrule
            6     &     0   \\ 
            56    &     2   \\ 
            106   &     4   \\
            156   &     6   \\
            206   &     8   \\
            256   &    10   \\
            306   &    12   \\
            356   &    14   \\
            406   &    16   \\
            456   &    19   \\
            506   &    21   \\
            556   &    23   \\
            606   &    25   \\
            656   &    27   \\
            706   &    29   \\
            756   &    31   \\
            806   &    33   \\
            856   &    35   \\
            906   &    38   \\
            956   &    40   \\
            1006  &    42   \\
            \bottomrule
        \end{tabular}
    \end{table}

    \FloatBarrier
