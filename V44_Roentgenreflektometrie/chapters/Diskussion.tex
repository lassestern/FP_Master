\newpage
\section{Diskussion}
    Bevor die Eigenschaften der Polystyrol- und Siliziumschicht untersucht werden konnten, mussten zunächst Eigenschaften des Röntgenstrahls sowie der Messgeometrie bestimmt werden. Die Bestimmung der 
    Strahlbreite aus dem z-Scan konnte erfolgreich durchgeführt werden und kann aufgrund der bereits annähernd perfekten Winkelausrichtung der Probe zum Strahl bei diesem Scan auch als zuverlässig angesehen 
    werden. Der zur Korrektur der Intensitäten wichtige Korrekturfaktor G wurde auf zwei Wegen bestimmt, die annähernd gleiche Ergebnisse lieferten:
    
    \begin{enumerate}
      \item $\overline{\alpha_{\text{g, grafisch}}} = \SI{0.64}{\degree}$
      \item $\alpha_{\text{g, math}} = \arcsin\left(\frac{d_0}{D}\right) = \SI{0.63}{\degree}$
    \end{enumerate}

    Die Korrektur in dem, durch diesen Winkel festgelegten, Bereich führt zu einer konstanten Reflektivität von eins unter dem kritischen Winkel, sodass die Bestimmung des Geometriewinkels und die 
    anschließende Korrektur der Reflektivität als erfolgreich angesehen werden können.\newline

    Die Anpassung der Reflektivität des Systems durch den Parratt-Algorithmus reproduziert die gemessene Reflektivität im Winkelbereich von 0,45° bis 0,90° sehr gut. Für kleinere Winkel liegt die angepasste
    Reflektivität etwas zu hoch und für höhere Winkel etwas zu niedrig. Die aus dem Algorithmus bestimmte Polystyrol-Schichtdicke und die aus der Periode der Oszillation bestimmte 
    
    \begin{enumerate}
      \item $z_{\text{PS}, \Delta\alpha} = \SI{865+-19}{\angstrom}$
      \item $z_{\text{PS, Parratt}} = \SI{866}{\angstrom}$
    \end{enumerate}

    gleichen sich beinahe exakt und können als gut bestimmt bezeichnet werden. Es ist einerseits zu erwähnen, dass der Ausgangswert der Schichtdicke bei der Anpassung des Parratt-Algorithmus auf den zuvor
    berechneten Wert gesetzt worden ist, dieser aber auch gute Reflektivitätswerte liefert. Die Dispersionen 
    
    \begin{enumerate}
      \item $\delta_{\text{PS}}=\SI{9.000e-7}{} \qquad \delta_{\text{PS, Lit}}=\SI{3.5e-6}{} \qquad  A=\SI{74.3}{\percent}$
      \item $\delta_{\text{Si}}=\SI{6.840e-6}{} \qquad \delta_{\text{Si, Lit}}=\SI{7.6e-6}{} \qquad  A=\SI{10.0}{\percent}$
    \end{enumerate}
    
    wurden ebenfallsüber den Parratt-Algorithmus bestimmt und geben ein unterschiedliches Bild wieder. Die Dispersion des Siliziums hatte nur einen geringen Einfluss auf die Anpassung und Werte nahe dem 
    Literaturwert ergaben eine gute Anpassung. Die Dispersion des Polystyrols hingegen musste deutlich kleiner als der Literaturwert gewählt werden, um die Amplitude der Kiessig-Oszillationen 
    gut zu reproduzieren.
    Die aus den Dispersionen bestimmten kritischen Winkel

    \begin{enumerate}
      \item $\alpha_{\text{c, PS}} = \SI{0.0769}{\degree} \qquad \alpha_{\text{c, PS}} = \SI{0.153}{\degree} \qquad A=\SI{49.7}{\percent}$
      \item $\alpha_{\text{c, Si}} = \SI{0.2119}{\degree} \qquad \alpha_{\text{c, Si}} = \SI{0.223}{\degree} \qquad A=\SI{5.0}{\percent}$
    \end{enumerate}

    spiegeln das Bild der Dispersionen wieder. Während der kritische Winkel des Siliziums nahe dem Literaturwert liegt und mit einem Intensitätsabfall zusammenfällt, tritt bei dem kritischen Winkel des
    Polystyrols eine deutlich größere Abweichung auf. Dass dieser Winkel nicht mit einem starken Intensitätsabfall zusammenfällt, stellt keinen Widerspruch dar, da Totalreflexion an der tieferliegenden
    Siliziumschicht die gesamte Reflektivität annähernd konstant hält.
    Die auch aus der Anpassung des Parratt-Algorithmus bestimmten Rauigkeiten 

    \begin{enumerate}
      \item $\sigma_1=\SI{1.080e-9}{1\per\angstrom}$
      \item $\sigma_2=\SI{7.056e-10}{1\per\angstrom}$
    \end{enumerate}

    haben keine literarischen Vergleichswerte. Da die beiden Werte jedoch für die feine Anpassung des Parratt-Algorithmus relevant waren, lassen sich die beiden Werte anhand der Güte der Anpassung eher 
    als ungefähre Schätzung auf die eigentlichen Werte der Rauigkeiten der vorliegenden Schichten bewerten. Alle hier genutzten Literaturwerte wurden aus \cite{tolan_x-ray_1999} entnommen.

    
    
